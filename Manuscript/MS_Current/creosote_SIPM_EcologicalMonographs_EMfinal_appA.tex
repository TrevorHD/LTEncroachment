\documentclass[11pt]{article}\usepackage[]{graphicx}\usepackage[usenames,dvipsnames]{xcolor}
% maxwidth is the original width if it is less than linewidth
% otherwise use linewidth (to make sure the graphics do not exceed the margin)
\makeatletter
\def\maxwidth{ %
  \ifdim\Gin@nat@width>\linewidth
    \linewidth
  \else
    \Gin@nat@width
  \fi
}
\makeatother

\definecolor{fgcolor}{rgb}{0.345, 0.345, 0.345}
\newcommand{\hlnum}[1]{\textcolor[rgb]{0.686,0.059,0.569}{#1}}%
\newcommand{\hlstr}[1]{\textcolor[rgb]{0.192,0.494,0.8}{#1}}%
\newcommand{\hlcom}[1]{\textcolor[rgb]{0.678,0.584,0.686}{\textit{#1}}}%
\newcommand{\hlopt}[1]{\textcolor[rgb]{0,0,0}{#1}}%
\newcommand{\hlstd}[1]{\textcolor[rgb]{0.345,0.345,0.345}{#1}}%
\newcommand{\hlkwa}[1]{\textcolor[rgb]{0.161,0.373,0.58}{\textbf{#1}}}%
\newcommand{\hlkwb}[1]{\textcolor[rgb]{0.69,0.353,0.396}{#1}}%
\newcommand{\hlkwc}[1]{\textcolor[rgb]{0.333,0.667,0.333}{#1}}%
\newcommand{\hlkwd}[1]{\textcolor[rgb]{0.737,0.353,0.396}{\textbf{#1}}}%
\let\hlipl\hlkwb

\usepackage{framed}
\makeatletter
\newenvironment{kframe}{%
 \def\at@end@of@kframe{}%
 \ifinner\ifhmode%
  \def\at@end@of@kframe{\end{minipage}}%
  \begin{minipage}{\columnwidth}%
 \fi\fi%
 \def\FrameCommand##1{\hskip\@totalleftmargin \hskip-\fboxsep
 \colorbox{shadecolor}{##1}\hskip-\fboxsep
     % There is no \\@totalrightmargin, so:
     \hskip-\linewidth \hskip-\@totalleftmargin \hskip\columnwidth}%
 \MakeFramed {\advance\hsize-\width
   \@totalleftmargin\z@ \linewidth\hsize
   \@setminipage}}%
 {\par\unskip\endMakeFramed%
 \at@end@of@kframe}
\makeatother

\definecolor{shadecolor}{rgb}{.97, .97, .97}
\definecolor{messagecolor}{rgb}{0, 0, 0}
\definecolor{warningcolor}{rgb}{1, 0, 1}
\definecolor{errorcolor}{rgb}{1, 0, 0}
\newenvironment{knitrout}{}{} % an empty environment to be redefined in TeX

\usepackage{alltt}
\usepackage[sc]{mathpazo} %Like Palatino with extensive math support
\usepackage{fullpage}
\usepackage[authoryear,sectionbib,sort]{natbib}
\linespread{1.7}
\usepackage[utf8]{inputenc}
\usepackage{lineno}
\usepackage{titlesec}
\titleformat{\section}[block]{\Large\bfseries\filcenter}{\thesection}{1em}{}
\titleformat{\subsection}[block]{\Large\itshape\filcenter}{\thesubsection}{1em}{}
\titleformat{\subsubsection}[block]{\large\itshape}{\thesubsubsection}{1em}{}
\titleformat{\paragraph}[runin]{\itshape}{\theparagraph}{1em}{}[. ]\renewcommand{\refname}{Literature Cited}
% my addnl packages
\usepackage{geometry}
\usepackage{graphicx}
\usepackage[T1]{fontenc}
\usepackage[utf8]{inputenc}
\usepackage{authblk}
\usepackage{setspace}
\usepackage{amsfonts,amssymb,amsmath,hyperref}
\usepackage{float}
\usepackage{caption}
\usepackage{multirow}
\usepackage{hyperref}
\usepackage{wrapfig}
\usepackage{rotating}
\usepackage[usenames,dvipsnames]{xcolor}
\newcommand{\revise}[1]{{\color{Mahogany}{#1}}}
\usepackage[normalem]{ulem}
\newcommand{\tom}[2]{{\color{red}{#1}}\footnote{\textit{\color{red}{#2}}}}

\doublespacing
%\bibliography{creosote_SIPM}



\title{Appendix S1}
\author{Trevor H., Drees, Brad M. Ochocki, Scott L. Collins, and T.E.X. Miller}
\date{\vspace{-5ex}}
\IfFileExists{upquote.sty}{\usepackage{upquote}}{}
\begin{document}
\maketitle
\noindent{} \textbf{Demography and dispersal at a grass-shrub ecotone: a spatial integral projection model for woody plant encroachment. \textit{Ecological Monographs}}

\renewcommand{\thefigure}{S\arabic{figure}}\setcounter{figure}{0}
\renewcommand{\thetable}{S\arabic{table}}\setcounter{table}{0}
\renewcommand{\theequation}{S\arabic{equation}}\setcounter{equation}{0}

\paragraph{WALD dispersal kernel}
In order to create the dispersal kernel, we first take the wind speeds at measurement height $z_{m}$ and correct them to find wind speed $U$ for any height $H$ by using the logarithmic wind profile

\begin{linenomath*} \begin{equation} U = \frac{1}{H} \int_{d+z_{0}}^{H} \frac{u^*}{K} \log \left(\frac{z-d}{z_{0}}\right) dz \end{equation} 
\end{linenomath*} 

given in \citet{bullock2012modelling} equation 6, with the notation slightly modified. 
Here, $z$ is the height above the ground, $K$ is the von Karman constant, and $u^*$ is the friction velocity.
The zero-plane displacement $d$ and roughness length $z_{0}$ are surface roughness parameters that, for a grass canopy height $h$ above the ground, are  approximated by $d \approx 0.7h$ and $z_{0} \approx 0.1h$.
These estimates are from \citet{raupach1994simplified} for a canopy area index $\Lambda = 1$ in which the sum of grass canopy elements is equal to the unit area being measured.
A 0.15 m grass height at our study site gives $d = 0.105$ and $z_{0} = 0.015$, which are suitable approximations for grassland \citep{wiernga1993representative}.
Calculations of $u^*$ were done using equation A2 from \citet{skarpaas2007dispersal}, in which 

\begin{linenomath*} \begin{equation} u^* = KU_{m} \left[\log\left(\frac{z_{m} - d}{z_{0}}\right)\right]^{-1} \end{equation} 
\end{linenomath*} 

and $U_{m}$ is the mean wind velocity at the measurement height $z_{m}$.
Values for the turbulent flow parameter $\sigma$ were then calculated using the estimate made by \citet{skarpaas2007dispersal} in their equation A4, where 

\begin{linenomath*} \begin{equation} \sigma = 2A_{w}^2 \sqrt{\frac{K(z-d)u^*}{C_{0}U}} \end{equation} 
\end{linenomath*} 

and $C_{0}$ is the Kolmogorov constant.
$A_{w}$ is a constant that relates vertical turbulence to friction velocity and is approximately equal to 1.3 under the assumptions of above-canopy flow made by \citet{skarpaas2007dispersal}, based off calculations from \citet{hsieh1997dissipation}.
We used maximum plant height $H$ as a measure of $z$.

The values from the previous three equations give us the necessary information to calculate $\mu'$ and $\lambda'$, thus allowing us to create the WALD distribution $p(r)$.
However, the base WALD model does not take into account variation in wind speeds or seed terminal velocities, which limits its applicability in systems where such variation is present.
In order to account for this variation, we integrate the WALD model over distributions of these two variables using the same method as \citet{skarpaas2007dispersal}.
Additionally, the WALD model assumes seed release from a single point source, which is not realistic for creosotebush; because seeds are released across the entire height of the shrub rather than from a point source, we integrated $p(r)$ across the uniform distribution from the grass canopy height to the shrub height.
Thus, under the assumptions that the height at which a seed is located does not affect its probability of being released and that seeds are evenly distributed throughout the shrub, this gives the dispersal kernel $K(r)$, where

\begin{linenomath*} \begin{equation} K(r) = \iiint p(F)p(U)p(z)p(r) \,dF\,dU\,dz \end{equation} 
\end{linenomath*} 

and $p(F)$ and $p(U)$ are the PDFs of the terminal velocity $F$ and wind speed $U$, respectively, and $p(z)$ is the uniform distribution from $h$ to $H$.

\paragraph{Dispersal data collection}
The distribution $p(F)$ in the integral above was constructed using experimentally determined seed terminal velocities.
These velocities were estimated using laboratory-based seed release experiments with a high-speed camera and motion tracking software to determine position as a function of time.
We then used the Levenberg-Marquardt algorithm to solve a quadratic-drag equation of motion for $F$. 
Before seeds were released, they were dried, dyed with yellow fluorescent powder, and then put against a black background to improve visibility and make tracking easier.
While the powder added mass to the seeds, this added mass only yielded an approximately 2.5\% increase, likely having little effect on terminal velocities.
Measurements were conducted for 48 seeds that were randomly chosen from a seed pool derived from different plants, and then an empirical PDF of terminal velocities was constructed using the data.
Constructing $p(U)$ involved creating an empirical PDF of hourly wind speeds using data from Sevilleta LTER meterological station 49, the station closest to our transects.
We used wind speed data collected hourly from 2015 to 2019 \citep{SEVmet}.

\bibliographystyle{ecology}
\bibliography{creosote_SIPM}

\end{document}
